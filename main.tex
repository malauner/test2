% KOMA-Script für professionelle wissenschaftliche Arbeiten
\documentclass[10pt,a4paper,twoside,openright,bibliography=totoc]{scrartcl}

% Encoding und Sprache
\usepackage[utf8]{inputenc}
\usepackage[german]{babel}
\usepackage[T1]{fontenc}

% Schriftart Helvetica (serifenlos)
\usepackage{helvet}
\renewcommand\familydefault{\sfdefault}

% Seitenränder für wissenschaftliche Arbeiten optimiert
\usepackage[left=2.5cm,right=2.5cm,top=2.5cm,bottom=2.5cm,bindingoffset=0.5cm]{geometry}

% Grafiken und PDF-Einbindung
\usepackage{graphicx}
\usepackage{pdfpages}

% Mathematische Symbole (falls benötigt)
\usepackage{amsmath}
\usepackage{amsfonts}
\usepackage{amssymb}

% Tabellen verbessern
\usepackage{booktabs}
\usepackage{longtable}

% Bibliographie mit APA-Stil
\usepackage[style=apa,backend=biber,language=german]{biblatex}
\addbibresource{testbiblio.bib}

% Hyperlinks (sollte als letztes Package geladen werden)
\usepackage[colorlinks=true, linkcolor=blue, urlcolor=red, citecolor=green]{hyperref}

% Metadaten für das Dokument
\author{Marc C., Christian H.}
\title{Enterprise Architecture}
\subtitle{Transferarbeit CAS EA3}
\date{\today}

\begin{document}

% Deckblatt aus HSLU-Vorlage einbinden
\includepdf[pages=1]{Vorlage_Deckblatt_Transferarbeit}

% Optional: Eigenes Titelbild/Logo auf separater Seite
\newpage
\begin{center}
\includegraphics[scale=0.08]{bilder/IMG_6461.JPG}
\end{center}

% Inhaltsverzeichnis
\newpage
\tableofcontents
\listoffigures % Optional: Abbildungsverzeichnis
\listoftables  % Optional: Tabellenverzeichnis

% Beispiel-Zitation (kann später entfernt werden)
\newpage
\section{verschiedene Zitatmöglichkeiten}
\textit{BibLaTex} mit \textit{biber} bietet diverse Möglichkeiten ein Zitat zu erfassen. Das Cheatsheet\footnote{\url{https://tug.ctan.org/info/biblatex-cheatsheet/biblatex-cheatsheet.pdf}} zeigt ein paar Möglichkeiten auf.

\subsection{Seitenzahlen}
Grundsätzlich kann die Seitenzahlen immer optional in eckigen Klammern mitgegeben werden. Grundsätzlich ist man dort ziemlich frei.

\subsection{cite - Autor, Jahr}
Mit dem Befehl \textit{cite} wird das Ganze als Text ausgegeben wie \cite[4]{haberli_ernuchternde_2019}.

\subsection{parencite - (Autor, Jahr)}
Mit dem Befehl \textit{parencite} mit alles in Klammer \parencite[2]{haberli_ernuchternde_2019} erzeugt.

\subsection{footcite - Fussnote mit Autor, Jahr}
Mit dem Befehl \textit{footcite} wird das Ganze als Fussnote\footcite[45]{haberli_ernuchternde_2019}.

\subsection{footcitetext - Fussnote nicht brauchen}
Mit dem Befehl \textit{footcitetext} erzeugt auch eine Fussnote. Jedoch wird die Nummerierung nicht korrekt erzeugt. Deshalb würde ich das nicht einsetzen.

\subsection{textcite - Autor (Jahr)}
Mit dem Befehl \textit{textcite} wird ein Text gemäss \textcite[56-66]{haberli_ernuchternde_2019} in Klammer gesetzt.

\subsection{smartcite - Fussnote mit Autor, Jahr}
Mit dem Befehl \textit{smartcite} wird Text \smartcite[2]{haberli_ernuchternde_2019} wird derselbe Effekt erreicht wie mit \textit{footcite}

\subsection{cite* - Jahr}
Mit dem Befehl \textit{cite*} wird nur das Jahr ohne Autor als Text ausgegeben wie \cite*[3]{steinhoff_wie_2003}.

\subsection{parencite* - (Jahr)}
Mit dem Befehl \textit{parencite*} wird das Jahr in Klammer ausgegeben, ohne vorhergehenden Autor wie \parencite*[3]{steinhoff_wie_2003}.

\subsection{supercite - Autor, Jahr}
Mit dem Befehl \textit{supercite} werden Autor und Jahr ohne Klammer \supercite[3]{steinhoff_wie_2003} erzeugt, anlog \textit{cite}.


\subsection{autocite}
Mit \textit{autocite} wird ein Zitat ohne Klammer erstellt wie mit \autocite[3]{steinhoff_wie_2003}. Ich könnte mir jedoch vorstellen, dass bei eineer anderen biblatex-Einstellung etwas anderes angezeigt wird. Deshalb eher mit Vorsicht zu geniessen. \\

\subsection{citeauthor - Autor}
Mit \textit{citeauthor} wird ein Zitat nur mit dem Autor angegeben, wie es \citeauthor{grimm_kinder-_1812} in ihrem Buch dargelegt haben.

\subsection{citeauthor* - Autor}
Mit \textit{citeauthor*} wird ein Zitat nur mit dem Autor angegeben, wie es \citeauthor*{grimm_kinder-_1812} in ihrem Werk bereits zum Besten gegeben haben. Diese scheint genau dasselbe zu sein wie oben.

\subsection{citetitle - Titel}
Mit \textit{citetitle} wird bei einem Zitat auf den Titel Bezug genommen, wie es im Buch \citetitle[3]{grimm_kinder-_1812} erwähnt wird.

\section{Erstellung Dokument inkl. Bibliographie}
Mit dem Befehl \textit{pdflatex dateiname} wird das Ganze normal erstellt und mit \textit{biber dateiname} wird für die Bibliographie benötigt. \textit{dateiname} sollte ohne Endung eingegeben werden, weil \textit{biber} damit ein Problem hat.

\subsection*{Lustiges}
\TeX ist eine schöne Darstellung wie \LaTeX . Ist dasselbe auch mit BibTex möglich?

% ============================================================
% HAUPTTEIL DER ARBEIT
% ============================================================

\newpage
\section{Einleitung}
\label{sec:einleitung}

% Hier: Motivation, Problemstellung, Zielsetzung, Abgrenzung

\subsection{Ausgangslage und Motivation}
\label{subsec:motivation}
tbd

\subsection{Zielsetzung der Arbeit}
\label{subsec:zielsetzung}
tbd

\subsection{Aufbau der Arbeit}
\label{subsec:aufbau}
tbd

\section{Grundlagen Enterprise Architecture}
\label{sec:grundlagen}

% Hier: Definitionen, Frameworks (TOGAF, Zachman, etc.)

\subsection{Definition und Abgrenzung}
\label{subsec:definition}
tbd

\subsection{Enterprise Architecture Frameworks}
\label{subsec:frameworks}
tbd

\subsection{Relevante Konzepte und Methoden}
\label{subsec:konzepte}
tbd

\section{Methodik und Vorgehen}
\label{sec:methodik}

% Hier: Wie wurde vorgegangen? Welche Methoden wurden angewendet?

tbd

\section{Analyse und Ergebnisse}
\label{sec:analyse}

% Hier: Hauptteil der Arbeit - Analyse, Konzepte, Lösungen

\subsection{Ist-Analyse}
\label{subsec:ist}
tbd

\subsection{Soll-Konzept}
\label{subsec:soll}
tbd

\subsection{Bewertung und Empfehlungen}
\label{subsec:bewertung}
tbd

\section{Diskussion}
\label{sec:diskussion}

% Hier: Interpretation der Ergebnisse, Limitationen, kritische Reflexion

tbd

\section{Fazit und Ausblick}
\label{sec:fazit}

% Hier: Zusammenfassung, Beantwortung der Forschungsfrage, Ausblick

tbd

% ============================================================
% ANHANG
% ============================================================

\newpage
\appendix
\section{Anhang}
\label{sec:anhang}

% Hier können zusätzliche Materialien eingefügt werden
% z.B. Interviewleitfäden, detaillierte Diagramme, Code-Listings

% ============================================================
% LITERATURVERZEICHNIS
% ============================================================

\newpage
\printbibliography

\end{document}
